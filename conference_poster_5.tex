%%%%%%%%%%%%%%%%%%%%%%%%%%%%%%%%%%%%%%%%%
% a0poster Landscape Poster
% LaTeX Template
% Version 1.0 (22/06/13)
%
% The a0poster class was created by:
% Gerlinde Kettl and Matthias Weiser (tex@kettl.de)
% 
% This template has been downloaded from:
% http://www.LaTeXTemplates.com
%
% License:
% CC BY-NC-SA 3.0 (http://creativecommons.org/licenses/by-nc-sa/3.0/)
%
%%%%%%%%%%%%%%%%%%%%%%%%%%%%%%%%%%%%%%%%%

%----------------------------------------------------------------------------------------
%	PACKAGES AND OTHER DOCUMENT CONFIGURATIONS
%----------------------------------------------------------------------------------------

\documentclass[a0,landscape]{a0poster}

\usepackage{multicol} % This is so we can have multiple columns of text side-by-side
\columnsep=100pt % This is the amount of white space between the columns in the poster
\columnseprule=3pt % This is the thickness of the black line between the columns in the poster

\usepackage[svgnames]{xcolor} % Specify colors by their 'svgnames', for a full list of all colors available see here: http://www.latextemplates.com/svgnames-colors

\usepackage{times} % Use the times font
%\usepackage{palatino} % Uncomment to use the Palatino font

\usepackage{graphicx} % Required for including images
\graphicspath{{figures/}} % Location of the graphics files
\usepackage{booktabs} % Top and bottom rules for table
\usepackage[font=small,labelfont=bf]{caption} % Required for specifying captions to tables and figures
\usepackage{amsfonts, amsmath, amsthm, amssymb} % For math fonts, symbols and environments
\usepackage{wrapfig} % Allows wrapping text around tables and figures
\usepackage{mathrsfs}
\usepackage{tikz-cd}
\usepackage{here}
\usepackage{mathtools}
\renewcommand{\theenumi}{\arabic{enumi}}
\renewcommand{\labelenumi}{(\theenumi) }
%% Theorems
\theoremstyle{plain}

\newtheorem{theorem}{Theorem}[section]
\newtheorem{conjecture}[theorem]{Conjecture}
\newtheorem{lemma}[theorem]{Lemma}
\newtheorem{proposition}[theorem]{Proposition}
\newtheorem{corollary}[theorem]{Corollary}

\theoremstyle{definition}
\newtheorem{definition}[theorem]{Definition}
\newtheorem{example}[theorem]{Example}
\newtheorem{remark}[theorem]{Remark}

%% Some operators
\DeclareMathOperator{\Hom}{\mathrm{Hom}}
\DeclareMathOperator{\Tor}{\mathrm{Tor}}
\DeclareMathOperator{\CHom}{\mathcal{H}\!\mathit{om}}
\DeclareMathOperator{\CTor}{\mathcal{T}\!\mathit{or}}
\DeclareMathOperator{\Auteq}{\mathrm{Auteq}}
\DeclareMathOperator{\Cone}{\mathrm{Cone}}
\DeclareMathOperator{\ev}{\mathrm{ev}}
\DeclareMathOperator{\id}{\mathrm{id}}
\DeclareMathOperator{\depth}{\mathrm{depth}}
\DeclareMathOperator{\Pic}{\mathrm{Pic}}
\DeclareMathOperator{\MCG}{\mathrm{MCG}}
\DeclareMathOperator{\PMCG}{\mathrm{PMCG}}
\DeclareMathOperator{\RHom}{\mathrm{RHom}}
\DeclareMathOperator{\Ker}{\mathrm{Ker}}
\DeclareMathOperator{\Image}{\mathrm{Im}}
\DeclareMathOperator{\Aut}{\mathrm{Aut}}
\DeclareMathOperator{\Inn}{\mathrm{Inn}}
\DeclareMathOperator{\Out}{\mathrm{Out}}
\DeclareMathOperator{\Supp}{\mathrm{Supp}}
\DeclareMathOperator{\SL}{\mathrm{SL}}
\DeclareMathOperator{\Spec}{\mathrm{Spec}}
\DeclareMathOperator{\Perf}{\mathrm{Perf}}
\DeclareMathOperator{\NS}{\mathrm{NS}}
\DeclareMathOperator{\Ext}{\mathrm{Ext}}
\DeclareMathOperator{\Hilb}{\mathrm{Hilb}}

\newcommand{\nc}{\newcommand}

%% Calligraphic letters

\nc{\cF}{{\mathcal{F}}}
\nc{\cG}{{\mathcal{G}}}
\nc{\cH}{{\mathcal{H}}}

\nc{\cO}{{\mathcal{O}}}
\nc{\cU}{{\mathcal{U}}}
\nc{\cW}{{\mathcal{W}}}

%% Blackboard letters
\nc{\bA}{{\mathbb{A}}}
\nc{\bC}{{\mathbb{C}}}
\nc{\bP}{{\mathbb{P}}}
\nc{\bQ}{{\mathbb{Q}}}
\nc{\bZ}{{\mathbb{Z}}}

%% Script letters
\nc{\sT}{{\mathscr{T}}}


\begin{document}

%----------------------------------------------------------------------------------------
%	POSTER HEADER 
%----------------------------------------------------------------------------------------

% The header is divided into three boxes:
% The first is 55% wide and houses the title, subtitle, names and university/organization
% The second is 25% wide and houses contact information
% The third is 19% wide and houses a logo for your university/organization or a photo of you
% The widths of these boxes can be easily edited to accommodate your content as you see fit

\begin{minipage}{\linewidth}
    \centering{
        \veryHuge \color{NavyBlue} \textbf{Half-spherical twists on derived categories of coherent sheaves} \color{Black}\\ % Title
        \Huge\text{based on arXiv:2302.12501}\\[1cm] % Subtitle
        \huge \textbf{Hayato Arai} \\ % Author(s)
        \huge  Graduate School of Mathematical Sciences,
        The University of Tokyo\\ % University/organization
        email : hayato@ms.u-tokyo.ac.jp
    }
\end{minipage}

% %
% \begin{minipage}[b]{0.25\linewidth}
%     \color{DarkSlateGray}\Large \textbf{Contact Information:}\\
%     Department Name\\ % Address
%     University Name\\
%     123 Broadway, State, Country\\\\
%     Phone: +1 (000) 111 1111\\ % Phone number
%     Email: \texttt{john@LaTeXTemplates.com}\\ % Email address
% \end{minipage}
% %
% \begin{minipage}[b]{0.19\linewidth}
%     \includegraphics[width=20cm]{logo.png} % Logo or a photo of you, adjust its dimensions here
% \end{minipage}

\vspace{2cm} % A bit of extra whitespace between the header and poster content

%----------------------------------------------------------------------------------------
\large
\begin{multicols}{3} % This is how many columns your poster will be broken into, a poster with many figures may benefit from less columns whereas a text-heavy poster benefits from more

    %----------------------------------------------------------------------------------------
    %	ABSTRACT
    %----------------------------------------------------------------------------------------

    % \color{Navy} % Navy color for the abstract

    % \begin{abstract}
    %     this is abstract
    % \end{abstract}

    %----------------------------------------------------------------------------------------
    %	INTRODUCTION
    %----------------------------------------------------------------------------------------

    \color{SaddleBrown} % SaddleBrown color for the introduction

    \section{Introduction}

    \color{DarkSlateGray} % DarkSlateGray color for the rest of the content
    \subsection{Mirror symmetry for singular fibers of type $\textrm{I}_n$ of elliptic surfaces}
    Let $\pi \colon S \to C$ be a relatively minimal, smooth projective elliptic surface.
    The possible singular fibers of $\pi$ are classified by Kodaira and N\'{e}ron.
    Among them, the singular fiber of type $\textrm{I}_n$ is the cyclic configuration of $n$ smooth rational curves.

    Lekili and Polishchuk \cite{MR3663596} established mirror symmetry between type $\textrm{I}_n$ singular fiber $Y_n$ and the $n$-punctured torus $T_n$, i.e.~ they showed that the derived category $D^b(Y_n)$ of coherent sheaves on $Y_n$ and the wrapped Fukaya category $D^\pi(\mathcal{W}(T_n))$ of $T_n$ are equivalent.

    For example, the equivalence includes the following correspondence of objects:
    \begin{center}
        \centering
        \begin{displaymath}
            \begin{tikzpicture}
                % fiber
                \draw[thick] (-9,-1)--(-3,8);
                \draw[thick] (-10,0)--(2,0);
                \draw[thick] (1,-1)--(-5,8);

                % points
                \filldraw[black] (-6, 3.5) circle (4pt);
                \filldraw[black] (-4, 0) circle (4pt);
                \filldraw[black] (-2, 3.5) circle (4pt);

                \draw(-6, 3.5) node[left]{$x_1$};
                \draw(-4, 0) node[below]{$x_2$};
                \draw(-2, 3.5) node[right]{$x_3$};

                % components
                \draw(-7, 2) node[left]{$G_1$};
                \draw(-2, 0) node[below]{$G_2$};
                \draw(-3, 5) node[right]{$G_3$};

                % big square
                \draw[dashed] (5,0)--(20,0);
                \draw[dashed] (5,0)--(5,8);
                \draw[dashed] (5,8)--(20,8);
                \draw[dashed] (20,0)--(20,8);

                % horizontal lines
                \draw[thick] (5, 1)--(20, 1);
                \draw[thick] (1, 2)--(3, 2);
                % \draw[thick] (3, 2)--(5, 2);
                % \draw[thick] (9, 2)--(11, 2);

                % \draw[thick] (0, 0.5)--(11, 0.5);

                % % vertical lines
                % \draw[thick] (2, 0)--(2, 4);
                % \draw[thick] (4, 0)--(4, 4);
                % \draw[thick] (10, 0)--(10, 4);

                % % punctures
                % \draw[dotted, thick] (5+1, 2)--(9-1, 2);
                % \foreach \u in {1, 3, 5, 9}
                %     {
                %         \filldraw[white] (\u, 2) circle (2pt);
                %         \draw[black] (\u, 2) circle (2pt);
                %     }

                % % notations
                % \draw(0, 0.5) node[left]{$\gamma_{\cO_Y}$};
                % \draw(2, 4) node[above]{$\gamma_{\cO_{x_1}}$};
                % \draw(4, 4) node[above]{$\gamma_{\cO_{x_2}}$};
                % \draw(10, 4) node[above]{$\gamma_{\cO_{x_n}}$};

                % \draw(2, 3) node[above left]{$\gamma_{\cO_{G_1}(-1)}$};
                % \draw(1, 3) to[out=-90,in=135](1.5, 2);
                % \draw(4, 3) node[above left]{$\gamma_{\cO_{G_2}(-1)}$};
                % \draw(3, 3) to[out=-90,in=135](3.5, 2);

                % \draw(11, 2) node[right]{$\gamma_{\cO_{G_n}(-1)}$};

            \end{tikzpicture}
        \end{displaymath}
    \end{center}


    Combining this with the theory of topological Fukaya categories of surfaces by Haiden, Katzarkov, and Kontsevich \cite{MR3735868}, Opper \cite{2020arXiv201108288O} described the autoequivalence group of $D^b(Y_n)$ with the following exact sequence:
    \begin{equation}
        1 \to (\bC^\times)^n \times \bZ[1] \times \Pic^0(Y_n) \to \Auteq{D^b(Y_n)} \xrightarrow{\Upsilon} \MCG(T_n) \to 1.
    \end{equation}
    Here $\MCG(T_n)$ denotes the mapping class group of $T_n$ and the morphism $\Upsilon$ is induced by the equivalence $D^b(C_n) \simeq D^\pi(\mathcal{W}(T_n))$.
    %----------------------------------------------------------------------------------------
    %	MATERIALS AND METHODS
    %----------------------------------------------------------------------------------------
    \subsection{Autoequivalences of elliptic surfaces}

    %Let $\pi \colon S \to C$ be a relatively minimal, smooth projective elliptic surface.
    Uehara \cite{MR3568337} gave the following description of the autoequivalence group $\Auteq D^b(S)$ of $D^b(S)$:
    \begin{theorem}
        \quad
        \begin{itemize}
            \item $S$ has non-zero Kodaira dimension
            \item all singular fibers of $\pi$ are non-multiple and of type $\rm{I}_n$, $n \geq 2$
            \item  $B = \langle T_{\cO_G(a)} \mid G \subset S \text{ : an irreducible component of a singular fiber, } a \in \bZ \rangle$ : the subgroup of $\Auteq D^b(S)$ generated by twist functors $T_{\cO_G(a)}$, where $\cO_G(a)$ is the line bundle of degree $a$ on $G \simeq \bP^1$
        \end{itemize}

        Then there is the exact sequence
        \begin{align}
            1 \to \langle B, (-)\otimes \cO_S(D)\mid D.F=0, F \textrm{ is a fiber } \rangle & \rtimes \Aut{S} \times \bZ[2]      \\
                                                                                            & \to \Auteq{D^b(S)} \to \SL(2,\bZ).
        \end{align}
    \end{theorem}
    \subsection{Main results}
    \begin{enumerate}
        \item In a general setting, we construct a ``restriction'' morphism $B \to \Auteq{D^b(F)}$ for each fiber $F$, which is nontrivial if $F$ is reducible.
        \item Combining with mirror symmetry for the singular fibers of type $\textrm{I}_n$, we described the group $B$ in terms of the mapping class group of the $n$-punctured torus. Then there exists the exact sequence
              \begin{equation}
                  1 \to \langle (-)\otimes \cO_S(Y_{n_j}) \mid j = 1, \dots, m\rangle \to B \xrightarrow{r} \prod_{j=1}^m \MCG(T_{n_j}),
              \end{equation}
              where $Y_{n_j}$ is the singular fiber of type $\textrm{I}_{n_j}$.
        \item For $G \subset Y_{n_j}$, $a \in \bZ$, and the curve $\gamma_{\cO_G(a)}$ on $T_{n_j}$ corresponding to $\cO_{G}(a)$ under the equivalence $D^b(C_n) \simeq D^\pi(\mathcal{W}(T_n))$, the twist functor $T_{\cO_G(a)}$ is mapped to the half twist along $\gamma_{\cO_G(a)}$.
        \item The image of $r$ is generated by the half twists along the finite number of curves $\{\gamma_{\cO_G}, \gamma_{\cO_G(-1)} \mid G \subset Y_{n_j}\text{: an irreducible component, } 1 \leq j \leq m \}$.
    \end{enumerate}

    \begin{center}
        \centering
        \begin{displaymath}
            \begin{tikzpicture}[scale=2.5]
                \draw (0, 1.5) circle[radius=1.5];
                \draw[thick] (-0.5, 1.5)--(0.5, 1.5);
                \draw(0, 1.5) node{$>$};
                \draw[dashed] (-1.5, 1.5)--(-0.5, 1.5);
                \draw[dashed] (0.5, 1.5)--(1.5, 1.5);

                \foreach \u in {-0.5, 0.5}
                    {
                        \filldraw[white] (\u, 1.5) circle (2pt);
                        \draw[black] (\u, 1.5) circle (2pt);
                    }
                \draw(0, 1.5) node[above]{$\gamma$};

                \draw(2.5, 1.5) node{$\rightarrow$};

                \draw (0+5, 1.5) circle[radius=1.5];
                \draw[thick] (-0.5+5, 1.5)--(0.5+5, 1.5);
                \draw(5, 1.5) node{$<$};


                \draw[dashed](-1.5+5, 1.5) to[out=-30,in=180](-0.5+5, 0.75);
                \draw[dashed](-0.5+5, 0.75) to[out=0,in=-90](0.5+5, 1.5);

                \draw[dashed](-0.5+5, 1.5) to[out=90,in=180](0.5+5, 2.25);
                \draw[dashed](0.5+5, 2.25) to[out=0,in=150](1.5+5, 1.5);


                \foreach \u in {-0.5, 0.5}
                    {
                        \filldraw[white] (\u+5, 1.5) circle (2pt);
                        \draw[black] (\u+5, 1.5) circle (2pt);
                    }
            \end{tikzpicture}
        \end{displaymath}
        \captionof{figure}{\color{Green} The half twist along the curve $\gamma$.}
    \end{center}


    \section{Sketch of the proof}
    \subsection{Result (1)}
    The result (1) is a consequence of the following theorem which generalizes the result of \cite{MR2200048}.
    \begin{theorem}
        \quad
        \begin{itemize}
            \item $\pi \colon X \to T$ : a flat morphism between smooth quasi-projective varieties
            \item $i \colon Y = \pi^{-1}(0) \hookrightarrow X$ : the fiber at $0 \in T$
            \item $E \in D^b(Y)$ : an object such that $i_* E \in D^b(X)$ is a spherical object
            \item $T_{i_* E} \in \Auteq D^b(X)$ : the corresponding twist functor
        \end{itemize}
        Then there is a unique $H_E \in \Auteq D^b(Y)$ which makes the following diagram commutative:
        \begin{equation} \label{eq:cd2}
            \begin{tikzcd}
                D^b(Y) \arrow[r,"i_*"]\arrow[d, "H_E"'] & D^b(X) \arrow[d,"T_{i_* E}"]\\
                D^b(Y) \arrow[r, "i_*"]& D^b(X).
            \end{tikzcd}
        \end{equation}
    \end{theorem}
    \subsection{Result (3)}
    \begin{enumerate}
        \item[(a)] There are correspondences between indecomposable objects of $D^b(Y_{n})$ and homotopy classes of curves on $T_{n}$ ($+$ additional data), and between dimensions of $\Hom$-spaces in $D^b(Y_{n})$ and intersection numbers of curves on $T_{n}$.
        \item[(b)] An element of $\MCG(T_{n})$ is determined by its action on $\pi_1(T_{n})$ (Dehn--Nielsen--Baer theorem).
    \end{enumerate}
    % \begin{theorem}
    %     \quad
    %     \begin{itemize}
    %         \item $\pi \colon S \to C$ : a relatively minimal, smooth projective elliptic surface
    %         \item all reducible fibers $\{Y_j\}_{j = 1}^m$ are non-multiple and of type $\textrm{I}$
    %         \item $n_j \geq 2$ : the number of irreducible components of $Y_j$
    %         \item $T_{n_j}$ : the $n_j$-punctured torus
    %     \end{itemize}
    %     Then there exists the exact sequence
    %     \begin{equation}
    %         1 \to \langle (-)\otimes \cO_S(Y_j) \mid j = 1, \dots, m\rangle \to B \to \prod_{j=1}^m \MCG(T_{n_j}).
    %     \end{equation}
    %     Moreover, the twist functors $T_{\cO_G(a)} \in B$ are mapped to the half twists along certain arcs on tori.
    % \end{theorem}
    % \section*{Materials and Methods}

    % Fusce magna risus, molestie ut porttitor in, consectetur sed mi. Vestibulum ante ipsum primis in faucibus orci luctus et ultrices posuere cubilia Curae; Pellentesque consectetur blandit pellentesque. Sed odio justo, viverra nec porttitor vel, lacinia a nunc. Suspendisse pulvinar euismod arcu, sit amet accumsan enim fermentum quis. In id mauris ut dui feugiat egestas. Vestibulum ac turpis lacinia nisl commodo sagittis eget sit amet sapien. Phasellus imperdiet, tortor vitae congue bibendum, felis enim sagittis lorem, et volutpat ante orci sagittis mi. Morbi rutrum laoreet semper. Morbi accumsan enim nec tortor consectetur non commodo nisi sollicitudin. Proin sollicitudin. Pellentesque eget orci eros. Fusce ultricies, tellus et pellentesque fringilla, ante massa luctus libero, quis tristique purus urna nec nibh. Proin sollicitudin. Pellentesque eget orci eros. Fusce ultricies, tellus et pellentesque fringilla, ante massa luctus libero, quis tristique purus urna nec nibh.


    % %------------------------------------------------

    % \subsection*{Main results}

    % Nulla vel nisl sed mauris auctor mollis non sed.

    % \begin{equation}
    %     E = mc^{2}
    %     \label{eqn:Einstein}
    % \end{equation}

    % Curabitur mi sem, pulvinar quis aliquam rutrum. (1) edf (2)
    % , $\Omega=[-1,1]^3$, maecenas leo est, ornare at. $z=-1$ edf $z=1$ sed interdum felis dapibus sem. $x$ set $y$ ytruem.
    % Turpis $j$ amet accumsan enim $y$-lacina;
    % ref $k$-viverra nec porttitor $x$-lacina.

    % Vestibulum ac diam a odio tempus congue. Vivamus id enim nisi:

    % \begin{eqnarray}
    %     \cos\bar{\phi}_k Q_{j,k+1,t} + Q_{j,k+1,x}+\frac{\sin^2\bar{\phi}_k}{T\cos\bar{\phi}_k} Q_{j,k+1} &=&\nonumber\\
    %     -\cos\phi_k Q_{j,k,t} + Q_{j,k,x}-\frac{\sin^2\phi_k}{T\cos\phi_k} Q_{j,k}\label{edgek}
    % \end{eqnarray}
    % and
    % \begin{eqnarray}
    %     \cos\bar{\phi}_j Q_{j+1,k,t} + Q_{j+1,k,y}+\frac{\sin^2\bar{\phi}_j}{T\cos\bar{\phi}_j} Q_{j+1,k}&=&\nonumber \\
    %     -\cos\phi_j Q_{j,k,t} + Q_{j,k,y}-\frac{\sin^2\phi_j}{T\cos\phi_j} Q_{j,k}.\label{edgej}
    % \end{eqnarray}

    % Nulla sed arcu arcu. Duis et ante gravida orci venenatis tincidunt. Fusce vitae lacinia metus. Pellentesque habitant morbi. $\mathbf{A}\underline{\xi}=\underline{\beta}$ Vim $\underline{\xi}$ enum nidi $3(P+2)^{2}$ lacina. Id feugain $\mathbf{A}$ nun quis; magno. Fusce convallis rutrum turpis, quis aliquet enim accumsan id. Vestibulum ullamcorper porttitor convallis. Integer sagittis interdum malesuada. Class aptent taciti sociosqu ad litora torquent per conubia nostra, per inceptos himenaeos. Sed adipiscing tristique orci at ullamcorper. Morbi accumsan, urna et porttitor pulvinar, lacus risus dignissim massa. Proin sollicitudin. Pellentesque eget orci eros. Fusce ultricies, tellus et pellentesque fringilla, ante massa luctus libero, quis tristique purus urna nec nibh.

    % %----------------------------------------------------------------------------------------
    % %	RESULTS 
    % %----------------------------------------------------------------------------------------

    % \section*{Results}

    % Donec faucibus purus at tortor egestas eu fermentum dolor facilisis. Maecenas tempor dui eu neque fringilla rutrum. Mauris \emph{lobortis} nisl accumsan. Aenean vitae risus ante. Pellentesque condimentum dui. Etiam sagittis purus non tellus tempor volutpat. Donec et dui non massa tristique adipiscing.
    % %
    % \begin{wraptable}{l}{12cm} % Left or right alignment is specified in the first bracket, the width of the table is in the second
    %     \begin{tabular}{l l l}
    %         \toprule
    %         \textbf{Treatments} & \textbf{Response 1} & \textbf{Response 2} \\
    %         \midrule
    %         Treatment 1         & 0.0003262           & 0.562               \\
    %         Treatment 2         & 0.0015681           & 0.910               \\
    %         Treatment 3         & 0.0009271           & 0.296               \\
    %         \bottomrule
    %     \end{tabular}
    %     \captionof{table}{\color{Green} Table caption}
    % \end{wraptable}
    % %
    % Phasellus imperdiet, tortor vitae congue bibendum, felis enim sagittis lorem, et volutpat ante orci sagittis mi. Morbi rutrum laoreet semper. Morbi accumsan enim nec tortor consectetur non commodo nisi sollicitudin. Proin sollicitudin. Pellentesque eget orci eros. Fusce ultricies, tellus et pellentesque fringilla, ante massa luctus libero, quis tristique purus urna nec nibh.

    % Nulla ut porttitor enim. Suspendisse venenatis dui eget eros gravida tempor. Mauris feugiat elit et augue placerat ultrices. Morbi accumsan enim nec tortor consectetur non commodo. Pellentesque condimentum dui. Etiam sagittis purus non tellus tempor volutpat. Donec et dui non massa tristique adipiscing. Quisque vestibulum eros eu. Phasellus imperdiet, tortor vitae congue bibendum, felis enim sagittis lorem, et volutpat ante orci sagittis mi. Morbi rutrum laoreet semper. Morbi accumsan enim nec tortor consectetur non commodo nisi sollicitudin.

    % \begin{center}\vspace{1cm}
    %     \includegraphics[width=0.8\linewidth]{placeholder}
    %     \captionof{figure}{\color{Green} Figure caption}
    % \end{center}\vspace{1cm}

    % In hac habitasse platea dictumst. Etiam placerat, risus ac.

    % Adipiscing lectus in magna blandit:

    % \begin{center}\vspace{1cm}
    %     \begin{tabular}{l l l l}
    %         \toprule
    %         \textbf{Treatments} & \textbf{Response 1} & \textbf{Response 2} \\
    %         \midrule
    %         Treatment 1         & 0.0003262           & 0.562               \\
    %         Treatment 2         & 0.0015681           & 0.910               \\
    %         Treatment 3         & 0.0009271           & 0.296               \\
    %         \bottomrule
    %     \end{tabular}
    %     \captionof{table}{\color{Green} Table caption}
    % \end{center}\vspace{1cm}

    % Vivamus sed nibh ac metus tristique tristique a vitae ante. Sed lobortis mi ut arcu fringilla et adipiscing ligula rutrum. Aenean turpis velit, placerat eget tincidunt nec, ornare in nisl. In placerat.

    % \begin{center}\vspace{1cm}
    %     \includegraphics[width=0.8\linewidth]{placeholder}
    %     \captionof{figure}{\color{Green} Figure caption}
    % \end{center}\vspace{1cm}

    % %----------------------------------------------------------------------------------------
    % %	CONCLUSIONS
    % %----------------------------------------------------------------------------------------

    % \color{SaddleBrown} % SaddleBrown color for the conclusions to make them stand out

    % \section*{Conclusions}

    % \begin{itemize}
    %     \item Pellentesque eget orci eros. Fusce ultricies, tellus et pellentesque fringilla, ante massa luctus libero, quis tristique purus urna nec nibh. Phasellus fermentum rutrum elementum. Nam quis justo lectus.
    %     \item Vestibulum sem ante, hendrerit a gravida ac, blandit quis magna.
    %     \item Donec sem metus, facilisis at condimentum eget, vehicula ut massa. Morbi consequat, diam sed convallis tincidunt, arcu nunc.
    %     \item Nunc at convallis urna. isus ante. Pellentesque condimentum dui. Etiam sagittis purus non tellus tempor volutpat. Donec et dui non massa tristique adipiscing.
    % \end{itemize}

    % \color{DarkSlateGray} % Set the color back to DarkSlateGray for the rest of the content

    % %----------------------------------------------------------------------------------------
    % %	FORTHCOMING RESEARCH
    % %----------------------------------------------------------------------------------------

    % \section*{Forthcoming Research}

    % Vivamus molestie, risus tempor vehicula mattis, libero arcu volutpat purus, sed blandit sem nibh eget turpis. Maecenas rutrum dui blandit lorem vulputate gravida. Praesent venenatis mi vel lorem tempor at varius diam sagittis. Nam eu leo id turpis interdum luctus a sed augue. Nam tellus.

    %----------------------------------------------------------------------------------------
    %	REFERENCES
    %----------------------------------------------------------------------------------------


    \bibliographystyle{amsalpha} % Plain referencing style
    \bibliography{sample} % Use the example bibliography file sample.bib

    %----------------------------------------------------------------------------------------
    %	ACKNOWLEDGEMENTS
    %----------------------------------------------------------------------------------------


    %----------------------------------------------------------------------------------------

\end{multicols}
\end{document}